\documentclass[a4paper,titlepage]{article}

\usepackage[T1]{fontenc}

\usepackage[margin = 2.5cm, includeheadfoot]{geometry}
\usepackage{fancyhdr}

\usepackage{amsmath}
\usepackage{amsfonts}
\usepackage{amssymb}

\author{Nicolas Trüssel}
\title{Embedded Systems Summary}

\pagestyle{fancy}
\fancyhf{}
\fancyhead[L]{Embedded Systems Summary}
\fancyhead[C]{Nicolas Trüssel}
\fancyhead[R]{Page \thepage}

\setlength\parindent{0pt}

\newcommand{\set}[1]{\lbrace #1 \rbrace}
\DeclareMathOperator{\lcm}{lcm}

\begin{document}
\maketitle

\section{Miscellaneous}
Dependable means
\begin{itemize}
	\item Reliable (continue working correctly when it worked correctly before)
	\item Maintainable (recover from errors)
	\item Available	(probability that it's working)
	\item Safety
	\item Security
\end{itemize}

Efficient in terms of:
\begin{itemize}
	\item Energy
	\item Code size
	\item Memory consumption
	\item Run time
	\item Weight
	\item Cost
\end{itemize}

\section{Scheduling}

\subsection{Definitions}
\begin{itemize}
	\item $J = \set{j_1, j_2, \dots, j_n}$ is a set of tasks.
	\item $a_i$ or $r_i$ is the arrival / release time of task $i$.
	\item $d_i$ is the deadline of task $i$.
	\item $C_i$ is the total computation time of task $i$.
	\item $c_i(t)$ is the the remaining execution time of task $i$ at time $t$.
	\item $s_i$ is the start time of task $i$.
	\item $f_i$ is the finish time of task $i$.
	\item $L_i = f_i - d_i$ is the lateness of task $i$.
	\item $E_i = \max\left(0, L_i\right)$ is the exceeding time or tardyness of task $i$.
	\item $X_i = d_i - a_i - C_i$ is the laxity or slack of task $i$.
\end{itemize}

\subsection{Generic Time Triggered Cyclic Executive Scheduler}
Let $f$ denote the frame length, $P$ the full period, $D(k)$ the relative deadline of task $k$,
and $p(k)$ the period of task $k$ (how often it occurs). Then the following
conditions have to be satisfied:
\begin{itemize}
	\item $\forall k . f \leq p(k)$ (at most one execution within a frame)
	\item $P = \lcm_k\left(p(k)\right)$
	\item $\forall k . f \geq C_k$ (processes start and complete within single frame)
	\item $\forall k . 2f - \gcd\left(p(k),f\right) \leq D(k)$ (between release time and deadline of every task there is at least one frame boundary)
\end{itemize}

\subsection{Aperiodic Scheduling}
\begin{center}
	{\renewcommand{\arraystretch}{2}
	\begin{tabular}{|c|c|c|}
		\hline
							& Equal arrival, non-preemptive	& Arbitrary arrival, preemptive \tabularnewline\hline
		Independent Tasks	& EDD							& EDF							\tabularnewline\hline
		Dependent Tasks		& LDF							& EDF*							\tabularnewline\hline
	\end{tabular}}
\end{center}

\subsubsection{EDD}
Schedule the tasks in order of non-decreasing deadlines. This minimizes the
maximal lateness.

\subsubsection{EDF}
Always execute the task with the earliest absolute deadline.
Schedulability test:
\begin{equation*}
	\forall i \in [n] . t + \sum_{k = 1}^{i} c_k(t) \leq d_i
\end{equation*}

\subsubsection{LDF}
Among all tasks without successors select the task with the latest deadline. Put
it in a stack. Repeat until no more tasks. Now execute tasks as they are on the
stack.

\subsubsection{EDF*}
Modify arrival and deadline of each task and use EDF on modified tasks.
\begin{align*}
	r_j^* &= \max_{j}\left(r_j, \max_{i}\left(r_i^* + C_i | J_i \to J_j\right)\right) \\
	d_i^* &= \min_{i}\left(d_i, \min_{j}\left(d_j^* - C_j | J_i \to J_j\right)\right)
\end{align*}

\end{document}
