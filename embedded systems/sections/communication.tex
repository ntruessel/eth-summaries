\section{Communication}
\subsection{Random Access}
Best sending rate is $p = \frac{1}{n}$ (using maximization of
$p \left(1 - p\right)^{n-1}$) leads to utilization of $\frac{1}{\emath}$.

\subsection{TDMA}
Statically allocated slots, synchronization done by master.

\subsection{CSMA/CD}
Detect collitions, do exponential backoff of $\set{0, \dots 2^n - 1}$ for $n$th
collision (possibly with upper bound on n).

\subsection{Token Ring}
Pass token around, attach to token.

\subsection{CSMA/CA}
IMAGE HERE

\subsection{CSMA/CR}
Before message is sent, arbitration happens. During arbitration every client can
detect which one is allowed to send. During arbitration, every node sends its
unique id and checks whether the result on the bus matches the id. Low bits
override high bits, hence the one with the lowest id wins.

\subsection{FlexRay}
FlexRay uses a combination of TDMA and something close to CSMA/CA alternating,
first a static segment using TDMA, then a dynamic segment

\subsection{Bluetooth}
Frequency hopping in range $2402 + k$ MHz, $0 \leq k \leq 78$, 1600hops per
second. Organized in piconets with 1 master and at most 7 slaves, which can be
combined into scatternets. A node is a master in at most one piconet and can be
a slave in multiple piconets.

The master can only send in even time slots. After each slot, the frequency is
changed using a pseudorandom sequence based on the master's device address
(unique) and the phase of the master's system clock. Packets are either 1, 3, or
5 slots long, the frequency is only changed after the full packet is sent,
however it changes 1, 3, or 5 steps, depending on the packet length.

The initial setup between master and slave is done by the inquiry step, where
the master sends it s ID using a special channel sequence, the slave listens and
replies with its ID, clock and other information.

Then the connection is started with in the page mode. The master sends its and
the slaves address using a special channel sequence, the slave listens whether
it hears its address and answers with its own address. The master then sends a
frequency hop synchronization packet (FHS) to the slave which allows them to
synchronize and connnect.

A slave can be in multiple states, active, hold (not processing date packtes),
sniff (awaken in regular time intervals), and park (no connection but still
synchronized).
