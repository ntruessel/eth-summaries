\section{Models}
\subsection{Sequence Graph}
A sequence graph is a hierarchy of directed acyclic dependence graphs with
dedicated start and end nodes. There are two kinds of nodes: task nodes and
hierarchy nodes (CALL for module calls, BR for branches, LOOP for iterations).

\subsubsection{Examples}

\begin{multicols}{2}
	\begin{lstlisting}
x = a * b;
y = x * c;
z = a + b;
submodule(a, z);

void submodule(int m, int n) {
	p = m + n;
	q = m * n;
}
	\end{lstlisting}
	\columnbreak
	\begin{tikzpicture}[node distance = 1.8cm]
		\node[nop]		(0)						{NOP};
		\node[inner]	(1)	[below left of=0]	{$\times$};
		\node[inner]	(2) [below of=1]		{$\times$};
		\node[inner]	(3) [below right of=0]	{$+$};
		\node[inner]	(4)	[below of=3]		{CALL};
		\node[nop]		(5) [below right of=2]	{NOP};
		\node[inner]	(7)	[below right of=4]	{$+$};
		\node[nop]		(6)	[above right of=7]	{NOP};
		\node[inner]	(8) [below right of=6]	{$\times$};
		\node[nop]		(9)	[below right of=7]	{NOP};

		\path	(0) edge[dashed]				(1)
				(0) edge[dashed]				(3)
				(1) edge[->]					(2)
				(2) edge[dashed]				(5)
				(3) edge[->]					(4)
				(4) edge[dashed]				(5)
				(4) edge[dashed, ->]			(6)
				(6) edge[dashed]				(7)
				(6) edge[dashed]				(8)
				(7) edge[dashed]				(9)
				(8) edge[dashed]				(9)
				(9) edge[dashed, ->, bend left]	(4)
		;
	\end{tikzpicture}
\end{multicols}

\begin{lstlisting}
x = a * b;
y = x * c;
z = a + b;

if (z > 0) {
	p = m + n;
	q = m * n;
}
\end{lstlisting}

\begin{tikzpicture}[node distance = 1.8cm]
	\node[nop]		(0)							{NOP};
	\node[inner]	(1)	[below left of=0]		{$\times$};
	\node[inner]	(2) [below of=1]			{$\times$};
	\node[inner]	(3) [below right of=0]		{$+$};
	\node[inner]	(4)	[below of=3]			{CALL};
	\node[nop]		(5) [below right of=2]		{NOP};

	\node[inner]	(7)		[below right of=4]	{$+$};
	\node[nop]		(6)		[above right of=7]	{NOP};
	\node[inner]	(8)		[below right of=6]	{$\times$};
	\node[nop]		(9)		[below right of=7]	{NOP};

	\node[nop]		(11)	[right of=8]		{NOP};
	\node[nop]		(10)	[above of=11]		{NOP};
	\node[nop]		(12)	[below of=11]		{NOP};

	\path	(0)		edge[dashed]					(1)
			(0)		edge[dashed]					(3)
			(1)		edge[->]						(2)
			(2)		edge[dashed]					(5)
			(3)		edge[->]						(4)
			(4)		edge[dashed]					(5)
			(4)		edge[dashed, ->]				(6)
			(6)		edge[dashed]					(7)
			(6)		edge[dashed]					(8)
			(7)		edge[dashed]					(9)
			(8)		edge[dashed]					(9)
			(9)		edge[dashed, ->, bend left]		(4)
			(4)		edge[dashed, ->, bend left] 	(10)
			(10)	edge[dashed, ->]				(11)
			(11)	edge[dashed, ->]				(12)
			(12)	edge[dashed, ->, bend left=60]	(4)
	;
\end{tikzpicture}
