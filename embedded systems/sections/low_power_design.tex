\section{Low Power Design}
\begin{itemize}
	\item $\alpha$ switching activity
	\item $C_L$ load capacity
	\item $f$ clock frequency
	\item $V_{dd}$ supply voltage
	\item $V_T$ threshold voltage ($V_T \ll V_{dd}$)
\end{itemize}
\begin{align*}
	P &\sim \alpha C_L V_{dd}^2 f \\
	\tau &\sim C_L \frac{V_{dd}}{\left(V_{dd} - V{T}\right)^2} \\
	f_{max} &\sim V_{dd} \\
	E &\sim \alpha C_L V_{dd}^2 \underbrace{\left(\text{\#cycles}\right)}_{f \cdot t}
\end{align*}

\subsection{YDS Algorithm}
The YDS algorithm can be use to schadule tasks with dynamic voltage scaling such
that the engergy consuption is minimal.

We define
\begin{align*}
	V'([z, z']) &= \set{v_i \in V: z \leq a_i < d_i \leq z'} \\
	G([z, z']) &= \sum_{v_i in V'([z,z'])} \frac{C_i}{z' - z}
\end{align*}

Algorithm:
\begin{enumerate}
	\item Calculate intensities
	\item Schedule jobs in highest intensity interval using EDF, intensity as
		frequency
	\item Adjust arrival times and deadlines by excluding the interval, restart
\end{enumerate}

The online version continuously updates best schedule for all available tasks.
It has at most 27 times higher energy consumption.

\subsection{DPM}
With dpm, make sleep periods as long as possible, reduce the amount of on/off
transitions. Use YDS and round frequencies up.
