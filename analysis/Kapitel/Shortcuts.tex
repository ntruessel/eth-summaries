% !TeX root = ../Main.tex
\section{Shortcuts, Abschätzungen, Tricks}
	\begin{theorem}[Bernoulli]
		Für $x \in \mathbb{R}, n \in \mathbb{N}$ gilt:
		$$
		(1 + x)^n \leq 1 + nx
		$$
	\end{theorem}
	\begin{theorem}[Stirling]
		$$ n! \approx \sqrt{2 \pi n} \cdot \emath ^{-n} \cdot n^n $$
	\end{theorem}
	\begin{proofhelp}[Partialbruchzerlegung]
		\hfill 
		\begin{enumerate}
			\item Falls nötig mache Polynomdivision.
			\item Bestimme die Nullstellen des Nenners.
			\item Teile den Bruch in die Summen von einzelnen Brüchen mit unbekannten Zählern und den Nullstellen in den Nennern. Folgende Regeln sind zu beachten:
			\begin{enumerate}
				\item Eine Nullstelle mit Vielfachheit n erzeugt n Summanden, der Nenner des i-ten Bruches ist die i-te Potenz der einfachen Nullstelle.
				\item Unabhängig von ihrer Vielfachheit hat eine Nullstelle mit Grad 1 eine Konstante im Zähler, eine Nullstelle mit Grad 2 einen Term der Form $ax + b$.
			\end{enumerate}
			\item Löse das Gleichungssystem (Summe der Partialbrüche = Ursprünglicher Bruch) durch Koeffizientenvergleich oder Einsetzen der Nullstellen für x.
		\end{enumerate}
	\end{proofhelp}
	\begin{shortcut}[Werte von Quadraturzeln] \hfill \\
		$\sqrt{2} \approx 1.41421 $, $\sqrt{3} \approx 1.73205 $, $\sqrt{5} \approx 2.23607 $, $ \sqrt{6} \approx 2.44949 $, $ \sqrt{7} \approx 2.64575 $, $\sqrt{10} \approx 3.16228 $
	\end{shortcut}
	
	\begin{center}
	\begin{tabular}{|C{4cm}|C{4cm}|}
		\hline
		\rowcolor[gray]{0.9}
		\text{Funktion }f(x)				&	\text{Stammfunktion }F(x)			\tabularnewline\hline
		\frac{1}{x}						&	\log(\abs{x})						\tabularnewline\hline
		x^n, n \neq -1					&	\frac{1}{n + 1}x^{n+1}				\tabularnewline\hline
		nx^{n-1}							&	x^n									\tabularnewline\hline
		\sin x							&	-\cos x								\tabularnewline\hline
		\cos x							&	\sin x								\tabularnewline\hline
		\tan x							&	-\log\abs{\cos x}					\tabularnewline\hline
		1 + \tan^2 x						&	\tan x								\tabularnewline\hline
		\sin^2 x							&	\frac{1}{2}(x - \sin x \cdot \cos x)	\tabularnewline\hline
		\cos^2 x							&	\frac{1}{2}(x + \sin x \cdot \cos x) \tabularnewline\hline
		\frac{1}{\sqrt{1-x^2}}			&	\arcsin x							\tabularnewline\hline
		\frac{-1}{\sqrt{1-x^2}}			&	\arccos x							\tabularnewline\hline
		\frac{1}{1 + x^2}				&	\arctan x							\tabularnewline\hline
		\sinh x							&	\cosh x								\tabularnewline\hline
		\cosh x							&	\sinh x								\tabularnewline\hline
		1 - \tanh^2 x					&	\tanh x								\tabularnewline\hline
		\frac{1}{\sqrt{x^2+1}}			&	\arsinh x							\tabularnewline\hline
		\frac{1}{\sqrt{x^2-1}}, x > 1	&	\arcosh x							\tabularnewline\hline
		\frac{1}{1-x^2}, \abs{x} < 1		&	\artanh x							\tabularnewline\hline
		\frac{1}{1-x^2}, \abs{x} > 1		&	\arcoth x							\tabularnewline\hline
	\end{tabular}
	\end{center}