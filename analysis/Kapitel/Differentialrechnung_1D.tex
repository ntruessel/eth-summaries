% !TeX root = ../Main.tex
\section{Differentialrechung auf $\mathbb{R}$}
	\begin{definition}[Diffenenzierbarkeit] 
		Sei $\Omega \subset \mathbb{R}$ offen, $f: \Omega \to \mathbb{R}, x_0 \in \Omega$
		\begin{enumerate}
			\item f heisst \highlight{differenzierbar} an der Stelle $x_0$ falls 
			$$ \diff{f}{x}(x_0) := f^\prime (x_0) := \lim\limits_{x \to x_0} \frac{f(x)- f(x_0)}{x- x_0}$$ 
			existiert.
			\item $f = (f_1, \dots, f_n): \Omega \to \mathbb{R}^n$ heisst differenzierbar an der Stelle $x_0$ falls jede Komponentenfunktion $f_i$ an $x_0$ differenzierbar ist und $f^\prime(x_0) = (f_1^\prime(x_0), \dots , f_n^\prime(x_0))$
			\item $f$ heisst auf $\Omega$ differenzierbar, falls f an jeder Stelle $x_0 \in \Omega$ differenzierbar ist.
		\end{enumerate}
	\end{definition}
	\begin{proofhelp}
		Differenzierbarkeit $\Rightarrow$ Stetigkeit.
	\end{proofhelp}
	\begin{hint}
		Um Differenzierbarkeit zu zeigen, verwende die Definition und leite \highlight{nicht} einfach ab. \\
		Um stetige Differenzierbarkeit zu zeigen, leite ab und überprüfe auf Stetigkeit.
	\end{hint}
	\\[1em]
	\begin{theorem}[Rechenregeln]
		Sind $f,g: \Omega \to \mathbb{R}$ an der Stelle $x_0 \in \Omega$ differenzierbar, so sind auch $f + g, f \cdot g, f \circ g$ und falls $g(x_0) \neq 0$ auch $f/g$ an $x_0$ stetig.
	\end{theorem}
	\begin{theorem}[Summenregel]
		Seien $f,g: \Omega \to \mathbb{R}$ an $x_0 \in \Omega$ differenzierbar. Es gilt:
		$$ (f + g)^\prime(x_0) = f^\prime(x_0) + g^\prime(x_0) $$
	\end{theorem}
	\begin{theorem}[Produktregel]
		Seien $f,g: \Omega \to \mathbb{R}$ an $x_0 \in \Omega$ differenzierbar. Es gilt:
		$$ (f \cdot g)^\prime(x_0) = f^\prime(x_0) \cdot g(x_0) + f(x_0) \cdot g^\prime(x_0) $$
	\end{theorem}
	\begin{theorem}[Quotientenregel]
		Seien $f,g: \Omega \to \mathbb{R}$ an $x_0 \in \Omega$ differenzierbar und $g(x_0) \neq 0$. Es gilt:
		$$ (f / g)^\prime(x_0) = \frac{f^\prime(x_0) \cdot g(x_0) - f(x_0) \cdot g^\prime(x_0)}{(g(x_0))^2} $$
	\end{theorem}
	\begin{theorem}[Kettenregel]
		Seien $f: \Omega \to \mathbb{R}$ an $x_0 \in \Omega$, $g: \mathbb{R} \to \mathbb{R}$ an $f(x_0)$ differenzierbar. Es gilt:
		$$ (g \circ f)^\prime(x_0) = g^\prime(f(x_0)) \cdot f^\prime(x_0) $$
	\end{theorem}
	\begin{hint}[Ableitung von Potenzfunktionen]
		\hfill\\
		Um $f(x) = g(x)^{h(x)}$ abzuleiten, benütze folgende Umformung:
		$$ f(x) = \exp \left( h(x) \cdot \log (g(x)) \right) \quad\Rightarrow\quad f^\prime(x) =  \left( h^\prime(x) \log (g(x)) + h(x) \frac{g^\prime(x)}{g(x)}\right)g(x)^{h(x)}$$
	\end{hint}
	\\[1em]
		\begin{theorem}[Mittelwertsatz]
			Seien $ -\infty < a < b < \infty, f: [a,b] \to \mathbb{R}$ stetig und in $(a,b)$ differenzierbar. Dann existiert ein $\xi \in (a,b)$ mit
			$$ f^\prime(\xi) = \frac{f(b) - f(a)}{b - a} $$
		\end{theorem}
		\begin{corollary}[Bernoulli de l'Hôpital]
			Seien $f,g: [a,b] \to \mathbb{R}$ stetig und in $(a,b)$ differenzierbar mit $\forall x \in (a,b) \, : \, g^\prime(x) \neq 0$. Sind $f(a) = g(a) = 0$ oder beide Funktionen divergieren bestimmt (gegen $\infty$ oder $-\infty$) und
			$$ \lim\limits_{x \downarrow a} \frac{f^\prime(x)}{g^\prime(x)} $$
			existiert, so gilt
			$$ \lim\limits_{x \downarrow a} \frac{f(x)}{g(x)} = \lim\limits_{x \downarrow a} \frac{f^\prime(x)}{g^\prime(x)} $$
		\end{corollary}
		\begin{theorem}[Umkehrsatz]
			Sei $f: (a,b) \to \mathbb{R}$ differenzierbar mit $f^\prime > 0$ auf $(a,b)$ und seien
			$$ -\infty \leq c = \inf\limits_{a<x<b} f(x) < \sup\limits_{a<x<b} f(x) = d \leq \infty $$
			dann ist f bijektiv und die Umkehrfunktion $f^{-1}: (c,d) \to \mathbb{R}$ ist differenzierbar mit
			$$ \forall x \in (a,b) \, : \, \left(f^{-1}\right)^\prime(f(x)) = \left(f^\prime(x)\right)^{-1} $$
			bzw.
			$$ \forall y \in (c,d) \, : \, \left(f^{-1}\right)^\prime(y) =  \frac{1}{f^\prime(f^{-1}(y))}$$
		\end{theorem}
	\\[1em]
		\begin{definition}[Taylorpolynom]
			Das n-te Taylorpolynom an der Entwicklungsstelle $a$ ist definiert als
			$$ T_n f(x;a) = \sum\limits_{k = 0}^{n} \frac{f^{(k)}(a)}{k!}(x-a)^k = f(a) + \frac{f^\prime(a)}{1!}(x-a) + \frac{f^{\prime\prime}(a)}{2!}(x-a)^2 + \cdots + \frac{f^{(n)}(a)}{n!} (x- a)^n $$
		\end{definition}
		\begin{definition}[Taylor-Formel]
			Sei $f \in C^{m-1}([a,x])$ auf $(a,x)$ m-mal differenzierbar. Dann gibt es $\xi \in (a,x)$ mit
			$$ f(x) = T_{m-1}(x;a) + f^{(m)}(\xi)\frac{(x-a)^m}{m!} $$
			$$ R_n f(x;a) = f^{(n+1)}(\xi)\frac{(x-a)^{n+1}}{(n+1)!} $$
		\end{definition}
		\begin{proofhelp}[Restterm]
			Für den Restterm gilt:
			$$ \lim\limits_{x \to a} \frac{R_n f(x;a)}{(x-a)^n} = 0 $$
		\end{proofhelp}
		\begin{proofhelp}[Restterm nach Integralrechnung]
			$$R_n f(x;a) = \frac{1}{n!}\int\limits_{a}^{x}(x-t)^n f^{(n+1)}(t) \d t$$
		\end{proofhelp}
	\\[1em]
		\begin{theorem}[Extremalstellen]
			\hfill\\
			Sei $\Omega \subset \mathbb{R}, f: \Omega \to \mathbb{R}, f \in C^{\infty}(\Omega), x_0 \in \Omega$. Einer der folgenden Fälle tritt ein:
			\begin{enumerate}
				\item $\forall j \geq 1 \, : \, f^{(j)}(x_0) = 0$
				\item $m := 1 + \max \{ j : f^{(i)}(x_0) = 0, 1 \leq i \leq j \}$ d.h $f^{(m)}(x_0) \neq 0$ und $f^{(1)}(x_0) = \dots = f^{(m - 1)}(x_0) = 0$
				\begin{enumerate}
					\item m ist ungerade, dann ist $x_0$ keine Extremalstelle.
					\item m ist gerade und $f^{(m)}(x_0) > 0$, dann ist $x_0$ eine strikte lokale Minimalstelle.
					\item m ist gerade und $f^{(m)}(x_0) < 0$, dann ist $x_0$ eine strikte lokale Maximalstelle.
				\end{enumerate}
			\end{enumerate}
		\end{theorem}