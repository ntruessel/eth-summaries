% !TeX root = ../Main.tex
\section{Differentialgleichungen}
	\begin{definition}
		Eine \highlight{Differentialgleichung} ist eine Gleichung für eine Funktion von einer oder mehreren Variablen, in der auch Ableitungen dieser Funktion vorkommen.
	\end{definition}
		\begin{definition}[Lineare Differentialgleichung]
			Eine lineare DGL n-ter Ordnung hat die Gestalt $y^{(n)}(x) + a_{n-1}(x)y^{(n-1)}(x) + \dots + a_0(x) y(x) = b(x)$, mit $a_i(x), b(x)$ Funktionen. 
		\end{definition}
		\subsection{Homogene lineare DGL mit konstanten Koeffizienten}
			\begin{definition}[Charakteristisches Polynom]
				Das charakteristische Polynom der Gleichung $y^{(n)}(x) + a_{n-1}y^{(n-1)}(x) + \dots + a_0y(x) = 0$ (homogene lineare DGL) ist gegeben durch
				$$ p(t) = t^n + a_{n-1} t^{n-1} + \dots + a_0 $$
			\end{definition}
			\begin{proofhelp}
				Um homogene lineare DGL mit konstanten Koeffizienten zu lösen, suche die Nullstellen des charakteristischen Polynoms und schaue unten welche Nullstellen zu welchen Lösungen führen. 
				Die allgemeine Lösung ist die Linearkombination all dieser Lösungen. ($y_{allg}(x) = c_1 \cdot \emath^{\lambda_1 x} \cdots $)
				\begin{enumerate}
					\item Ist $\lambda \in \mathbb{R}$ eine k-fache Nullstelle, so sind $\emath^{\lambda x}, x^1 \cdot \emath^{\lambda x}, \dots, x^{k-1}\emath^{\lambda x}$ Lösungen der DGL.
					\item Sind $\lambda = \alpha + \imath \beta$ und $\overline{\lambda} = \alpha - \imath \beta $ mit $\beta \neq 0$ ein Paar konjugiert komplexer Nullstellen mit Vielfachheit k, so sind 
					$$ \emath^{\alpha x} \sin (\beta x), \quad \emath^{\alpha x} \cos (\beta x), $$
					$$ x^1 \cdot \emath^{\alpha x} \sin (\beta x), \quad x^1 \cdot \emath^{\alpha x} \cos (\beta x), $$
					$$ \vdots $$
					$$ x^{k-1} \cdot \emath^{\alpha x} \sin (\beta x), \quad x^{k-1} \cdot \emath^{\alpha x} \cos (\beta x) $$
					Lösungen der DGL. Beachte, dass die Koeffizienten von $x^{i} \cdot \emath^{\alpha x} \sin (\beta x)$ und $x^{i} \cdot \emath^{\alpha x} \cos (\beta x)$ verschieden sein dürfen.
				\end{enumerate}
			\end{proofhelp}
		\subsection{Inhomogene lineare DGL mit konstanten Koeffizienten}
			\begin{proofhelp}
				Um inhomogene lineare DGL mit konstanten Koeffizienten zu lösen, gehe so vor:
				\begin{enumerate}
					\item Löse die zugehörige homogene DGL und erhalte $y_{h}$.
					\item Mache einen Ansatz vom Typ der rechten Seite und erhalte $y_{s}$.
					\item Kombiniere die Lösung $y_{allg} = y_{h} + y_{s}$.
				\end{enumerate}
			\end{proofhelp}
			\begin{hint}[Ansatz der rechten Seite]
				Finde in der Tabelle den Ansatz. Dieser wird nun in die DGL eingesetzt und so die noch unbekannten Koeffizienten bestimmt.			
			\end{hint}
			\begin{center}
				\begin{tabular}{|C{5cm}|C{8cm}|}
					\hline
					\rowcolor[gray]{0.9}
					\text{Störfunktion}																												&	\text{Ansatz für }y_p														\tabularnewline\hline
					a \cdot \emath^{\mu x}																											&	b \cdot \emath^{\mu x}														\tabularnewline\hline
					a \cdot \sin	 (\nu x)		\linebreak		\fracsize b \cdot \cos ( \nu x)															&	c \cdot \sin (\nu x) + d \cdot \cos (\nu x)									\tabularnewline\hline
					a \cdot \emath^{\mu x} \cdot \sin	 (\nu x)		\linebreak		\fracsize b \cdot \emath^{\mu x} \cdot \cos ( \nu x)				&	\emath^{\mu x} \cdot (c \cdot \sin (\nu x) + d \cdot \cos (\nu x))			\tabularnewline\hline
					P_n(x)																															&	R_n(x)																		\tabularnewline\hline
					P_n(x) \emath^{\mu x} 																											&	R_n(x) \emath^{\mu x}														\tabularnewline\hline
					P_n(x) \cdot \emath^{\mu x} \cdot \sin	 (\nu x)		\linebreak		\fracsize Q_n(x) \cdot \emath^{\mu x} \cdot \cos ( \nu x)		&	\emath^{\mu x} \cdot (R_n(x) \cdot \sin (\nu x) + S_n(x) \cdot \cos (\nu x))	\tabularnewline\hline
				\end{tabular}
			\end{center}
			wobei $P_n(x), R_n(x), Q_n(x)$ und $S_n(x)$ Polynome von Grad $n$ sind.\\[1em]
			\begin{hint}
				Liegt eine Linearkombination der Störfaktoren vor, so hat man auch als Ansatz eine entsprechende Linearkombination zu wählen.
			\end{hint}
			\begin{hint}
				Falls $\lambda = \mu + \imath \nu$ eine m-fache Nullstelle des charakteristischen Polynomes ist, so muss man den Ansatz mit $x^m$ multiplizieren.
			\end{hint}
		\subsection{Lineare DGL erster Ordnung mit allgemeinen Koeffizienten}
			Die lineare DGL hat die allgemeine Form
			$$ y^\prime(x) = a(x)\cdot y + b(x) $$
			\begin{proofhelp}[Homogener Fall]
				Falls $y(x) \neq 0$ gilt $y(x) = \emath^{A(x)}\cdot K, \quad K \in \mathbb{R} - \{0\}$.\\ $A(x)$ ist eine Stammfunktion von $a(x)$.
			\end{proofhelp}
			\begin{proofhelp}[Allgemeiner Fall]
				Die allgemeine Lösung der DGL $ y^\prime(x) = a(x)\cdot y + b(x) $ lautet:
				$$ y_{all}(x) = \emath^{A(x)}\cdot \int \left( b(x) \emath^{-A(x)} \right) \d x + K \cdot \emath^{A(x)} $$
				wobei $K \in \mathbb{R}$ und $A(x)$ eine Stammfunktion von $a(x)$ ist.
			\end{proofhelp}
			\begin{hint}[Variation der Konstanten]
				Dies ist eine Alternative zum Lösungsschema im allgemeinen Fall. \\
				\begin{enumerate}
					\item Ignoriere $b(x)$ und löse den homogenen Fall. Erhalte $y_h(x)$
					\item Bestimme $y_p(x)$ so: $y_p(x) = y_h(x) \cdot \int \frac{b(x)}{y_h(x)} \d x \quad$ (setze $y_h$ mit $K = C(t)$ in die DGL ein).
					\item $y_{allg}(x) = y_h(x) + y_p(x)$
				\end{enumerate}
			\end{hint}
		\subsection{Separierbare DGL}
			\begin{hint}[Sparation der Variablen]
				Eine DGL der Form $y^\prime(x) = f(x)g(y)$ löst man durch Separation der Variablen, d.h. ersetze $y^\prime(x)$ durch $\diff{y}{x}$, integriere beide Seiten und löse nach $y$ auf. Möglicherweise ist hier auch eine Substitution sehr hilfreich.
			\end{hint}
		\subsection{Bernoulli DGL}
			\begin{proofhelp}[Sparation der Variablen]
				Eine DGL der Form $y^\prime(x) = f(x)y + g(x)y^{\alpha}$ löst man durch Substitution. Wähle 
				$$u = y^{1 - \alpha} \Leftrightarrow y = u^{\frac{1}{1-\alpha}}$$
				so erhält man
				$$ u^\prime = \underbrace{(1 - \alpha)f(x)}_{f^*(x)}u + \underbrace{(1 - \alpha) g(x)}_{g^*(x)} $$
				Dies löst man mit dem Ansatz für lineare DGL erster Ordnung mit allgemeinen Koeffizienten.
			\end{proofhelp}