% !TeX root = ../Formelsammlung.tex
\section{Thermodynamik}
\subsection{Druck}
\begin{equation}
p = \frac{F}{A} 
\end{equation}
\begin{equation}
\left[ p \right] = \mathrm{Pa}, \quad 1 \, \mathrm{Pa} = 1 \, \mathrm{N} / \mathrm{m^2}, \quad 1 \, \mathrm{bar} = 10^5 \, \mathrm{Pa}, \quad 1 \, \mathrm{atm} = 1,01325 \cdot 10^5 \, \mathrm{Pa}
\end{equation}
\subsection{Gesetz von Gay-Lussac}
\begin{equation}
V = C_1 \cdot T, \quad \text{bei konstantem Druck}, \quad \left( V \propto T \right)
\end{equation}
\subsection{Gesetz von Boyle und Mariotte}
\begin{equation}
p = C_2 \cdot T, \quad \text{bei konstantem Volumen}, \quad \left( p \propto T \right)
\end{equation}
\subsection{Zustandsgleichung des idealen Gases}
\begin{equation}
p \cdot V = n \cdot R \cdot T = N \cdot k \cdot T
\end{equation}
$p$ = Druck, $V$ = Volumen $n$ = Anzahl Mole, $N$ = Anzahl Moleküle, $T$ = absolute Temperatur
\begin{align}
k &= 1,381 \cdot 10^{-23} \mathrm{J / K} \qquad \text{(Boltzmann-Konstante)} \\ 
R &= N_Ak = 8,314 \frac{\mathrm{J}}{\mathrm{mol \cdot K}}
\end{align}
\subsection{Die Standardbedingungen}
\begin{align}
T &= 0^\circ C = 273,15 \text{ K} \\
p &= 1 \text{ atm}
\end{align}
\subsection{Wärmekapazität}
Die Wärmekapazität eines \textbf{Körpers} (als ganzes) ist definiert als
\begin{equation}
C = \frac{\Delta Q}{\Delta T}
\end{equation}
Die \textbf{speifische} Wärmekapazität ist 
\begin{equation}
c = \frac{\Delta Q}{m \Delta T}
\end{equation}
Die \textbf{molare} Wärmekapazität ist 
\begin{equation}
c = \frac{\Delta Q}{n \Delta T}
\end{equation}
Für die Erwärmung eines Körpers von $T_a$ auf $T_e$ wird die Wärmemenge (Energie) $Q$ benötigt:
\begin{equation}
Q = \int \d Q = \int \limits_{T_a}^{T_b} C \left( T \right) \d T
\end{equation}
Für kleine $\Delta T$ gilt
\begin{equation}
Q = C \cdot \Delta T
\end{equation}
\subsubsection{Wärmekapazität des idealen (einatomigen) Gases}
\begin{equation}
C = \frac{3}{2}Nk
\end{equation}
Für die molare Wärmekapazität gilt
\begin{equation}
c = \frac{3}{2}N_Ak = \frac{3}{2} R \approx 12,5 \frac{\mathrm{J}}{\mathrm{mol \cdot K}}
\end{equation}
\subsubsection{Wärmekapazität eines Festkörpers, Dulong-Petit}
Die spezifischen Wärmekapazitäten von Festkörpern variieren stark, die molaren Wärmekapazitäten sind bis auf einige Ausnahmen sehr ähnlich:
\begin{equation}
c \approx 25 \frac{\mathrm{J}}{\mathrm{mol \cdot K}}
\end{equation}
\subsection{Latente Wärme}
\begin{equation}
Q = m \cdot L, \quad \text{wobei L die spezifische latente Wärme und m die Masse ist}
\end{equation}
\subsection{Wärmestrahlung}
\subsubsection{Stefan-Bolzmannsches Gesetz}
Die (über alle Wellenlängen aufsummierte, auf der Fläche normierte und nach vorne abgestrahlte) Wärmestrahlung realer Körper ist
\begin{equation}
S \left( T \right) = \varepsilon \sigma T^4
\end{equation}
Dabei ist $\varepsilon \leq 1$ eine Zahl, die den Emissionsgrad des Körpers bezeichnet (oft temperaturabhängig, $\varepsilon = 1$ für schwarze Stahler) und $\sigma$ ist die Stefan-Bolzmann-Konstante:
\begin{equation}
\sigma = 5,670 \cdot 10^{-8} \frac{\mathrm{W}}{\mathrm{m^2 \cdot K^4}}
\end{equation}
Für einen Körper mit Temperatur $T$ bei einer Umgebungstemperatur $T_0$ gilt:
\begin{equation}
S_\mathrm{netto} = S_\mathrm{emmitiert} - S_\mathrm{absorbiert} = \varepsilon \sigma T^4 - \varepsilon \sigma {T_0}^4
\end{equation}
\subsubsection{Wiensches Verschiebungsgesetz}
\begin{equation}
\lambda_\mathrm{max} = \frac{2898 \mathrm{\mu m} \cdot K}{T}
\end{equation}
Wobei $\lambda_\mathrm{max}$ das Maximum der Spektralverteilungsfunktion $S \left( \lambda, T \right)$ ist (siehe unten).
\subsubsection{Spektralverteilungsfunkton}
Gesetz von Rayleigh-Jeans (Historisch und falsch!!):
\begin{equation}
S \left( \lambda, T \right) = \frac{2 \pi c}{\lambda^4}kT
\end{equation}
$k$ ist die Boltzmann-Konstante, $c$ die Lichtgeschwindigkeit \\[1em]
Gesetz von Plank:
\begin{equation}
S \left( \lambda, T \right) = \frac{2 \pi c^2 h}{\lambda^5} \frac{1}{e^{hc/\left(\lambda k T \right)} - 1}
\end{equation}
$k$ bezeichnet wiederum die Boltzmann-Konstante, $h$ ist die Planksche Konstante
\begin{equation}
h \approx 6,626 \cdot 10^{-34} \,\mathrm{J \cdot s}
\end{equation}
\subsection{Innere Energie}
Die innere Energie $U$ eines Körpers kann sowohl durch Wärmezufuhr als auch durch Leistung mechanischer Arbeit verändert werden. Es gilt:
\begin{equation}
\d U = \d Q + \d W
\end{equation}
\subsection{Mechanische Arbeit eines expandierenden Gases}
Ausgangszustand: Ein Gas mit Druck $p$ befindet sich in einem Behälter, der durch einen reibungsfreien Kolben verschlossen wird. Die vom Gas geleistete Arbeit bei einer Expansion (unabhängig von der Art der Expansion) um $\d V = A \d x$ ist:
\begin{equation}
\d W = -F \d x = - p A  \d x = - p \d V
\end{equation}
\subsection{Thermische Prozesse des idealen Gases}
Allgemein gilt:
\begin{equation}
W = \int\limits_{V_a}^{V_e} \d W = - \int\limits_{V_a}^{V_e}p \d V
\end{equation}
\subsubsection{Isobare Zustandsänderung}
\begin{equation}
W = - p \int\limits_{V_a}^{V_e} \d V = -p \left( V_e - V_e \right)
\end{equation}
\subsubsection{Isotherme Zustandsänderung}
Hier gilt: $$pV = konst.$$
Um die Temperatur des Gases bei der Expansion konstant zu halten, muss Energie zugeführt werden. Da $T$ konstant und die innere Energie $U$ eines idealen Gases nur von $T$ abhängt gilt:
\begin{equation}
\d U = \d Q + \d W = 0 \quad\Rightarrow\quad \d Q = - \d W
\end{equation}
Daraus folgt:
\begin{equation}
Q = - W = - \int\d W = \int\limits_{V_1}^{V_2}p \d V \,\overset{pV=nRT}{=}\, nRT \int\limits_{V_1}^{V_2}\frac{1}{V} \d V = nRT \ln\left(\frac{V_2}{V_1} \right)
\end{equation}
\subsubsection{Adiabatische Expansion}
Bei der adiabatischen Expansion wird keine Wärme ausgetauscht ($\d Q = 0$)
\begin{equation}
\d U - \d W = 0 \quad\Rightarrow\quad C \d T = - \frac{nRT}{V}\d V \quad\Rightarrow\quad \frac{\d T}{T} = - \frac{nR}{C}\frac{\d V}{V}
\end{equation}
\begin{equation}
\gamma \equiv 1 + \frac{nR}{C} \quad\Leftrightarrow\quad \gamma - 1 = \frac{nR}{C}
\end{equation}
\begin{equation}
\int \frac{1}{T}\d T = - \left(\gamma - 1\right) \int\frac{1}{V} \d V \quad\Rightarrow\quad \ln T + \left( \gamma - 1 \right) \ln V = konst.
\end{equation}
Daraus folgt:
\begin{equation}
TV^{\gamma - 1} = konst. \quad\wedge\quad \frac{pV}{nRT}V^{\gamma - 1} = konst. \Rightarrow pV^\gamma = konst.
\end{equation}
Arbeit:
\begin{equation}
\Delta W = \Delta U = c \cdot \Delta T, \quad \gamma = 1 + \frac{nR}{c} \quad \text{umformen und fertig.}
\end{equation}
\subsection{Wirkungsgrad}
Sei $Q_W$ die in warmen Reservoir aufgenommene Wärme, $Q_K$ die im kalten Reservoir abgegebene Wärme und $W = Q_W - Q_K$ die verrichtete Arbeit. Dann gilt für den Wirkungsgrad einer Wärmekraftmaschine:
\begin{equation}
\varepsilon = \frac{\abs{W}}{\abs{Q_W}} = 1 - \frac{\abs{Q_K}}{\abs{Q_W}}
\end{equation}
Im Falle einer carnotschen Wärmekraftmaschine (besser gehts nicht) gilt:
\begin{equation}
\varepsilon = 1 - \frac{T_3}{T_1}
\end{equation}
wobei $T_1$ die Temeperatur des warmen Reservoirs und $T_2$ die Temperatur des kalten Reservoirs ist.