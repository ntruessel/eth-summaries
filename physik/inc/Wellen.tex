% !TeX root = ../Formelsammlung.tex
\section{Wellen}
\subsection{Die Wellenfunkton}
Eine Welle ist von Ort und Zeitpunkt abhängig:
\begin{equation}
\xi = \xi (x, t)
\end{equation}
wobei Ortsabhängigkeit die Form der Welle und die Zeitabhängigkeit die Ausbreitung der Welle beschreiben. Für die Ausbreitungsgeschwindigkeit $v$ kann man auch folgende Gleichung formulieren ($+$ für Ausbreitung in negative x-Richtung, $-$ für Ausbreitung in positive x-Richtung): 
\begin{equation}
\xi (x, t) = \xi(t \pm vt)
\end{equation}
\subsection{Die harmonische Welle}
\begin{equation}
\xi (x, t) = \xi _0 \sin \left( k \left( x \pm vt \right) \right)
\end{equation}
wobei $k$ die Wellenzahl und $\xi _0$ die Amplitude ist. Weitere Beziehungen sind:
\begin{equation}
k = \dfrac{2 \pi}{\lambda}, \qquad f = \nu = \dfrac{\omega}{2 \pi} \quad\Leftrightarrow\quad \omega = 2 \pi \nu
\end{equation}
Somit kann man die Wellengleichung auch so formulieren:
\begin{equation}
\xi (x, t) = \xi _0 \sin \left( k x \pm \omega t  \right), \qquad \text{wobei} \enspace v = \dfrac{\omega}{k} = \nu \lambda
\end{equation}
\subsection{Die Wellengleichung}
\begin{equation}
\pdiff{^2 \xi}{t^2} - v^2 \pdiff{^2 \xi}{x^2} = 0
\end{equation}
Lösung:
\begin{equation}
\xi (x, t) = f(x -vt) - g(x + vt)
\end{equation}
\subsection{Seilwellen}
\begin{equation}
\text{Längendichte:} \quad \rho = \frac{M}{L} \quad\Rightarrow\quad \d m = \rho \d x
\end{equation}
\begin{equation}
v = \pm \sqrt{\frac{S}{\rho}}
\end{equation}
\subsection{Wellen in Festkörpern}
Die relative Längendehnung eines Stabes ist 
\begin{equation}
\varepsilon = \dfrac{\ell}{\Delta \ell} 
\end{equation}
Das Hookesche Gesetz lautet darauf angepasst:
\begin{equation}
F = Y \! A \varepsilon, \quad \text{wobei } Y \text{ das Elastizitätsmodul bezeichnet.}
\end{equation}
\subsubsection{Longitudinale elastische Welle}
Für die Ausbreitungsgeschwindigkeit $v$ der longitudinalen elastischen Welle in einem Festkörper gilt:
\begin{equation}
v = \sqrt{\dfrac{Y}{\rho}} 
\end{equation}
wobei $\rho$ hier die Volumendichte bezeichnet und $Y$ das oben erwähnte Elastizitätsmodul ist.
\subsection{Superposition harmonischer Wellen}
\begin{equation}
\xi (x_1, \Delta x, t) = \underbrace{2A \cos \left\{ \dfrac{1}{2} \left( \delta + k \Delta x \right) \right\}}_{\text{Amplitude}} \underbrace{ \sin \left\{ kx_1 - \omega t + \dfrac{1}{2} \left( \delta + k \Delta x \right) \right\}}_{\text{harmonische Welle}}
\end{equation}