% !TeX root = ../Formelsammlung.tex
\section{Energie und Arbeit}
\subsection{Energie}
\subsubsection{Energieerhaltung}
\begin{align}
\begin{split}
E_{tot} 	&= E_{Masse} + E_{kin} + E_{pot} + E_{chem} + \text{usw.} \\
		&= \text{konst.}
\end{split}
\intertext{aber für die Energie eines Körpers (ohne Reibung)}
E &= E_{kin} + E_{pot} \neq \text{konst.} 
\end{align}
\subsubsection{Geschwindigkeitsparameter}
\begin{equation}
\beta \equiv \dfrac{v}{c} = \frac{pc}{E} = \sqrt{1 - \frac{(mc^2)^2}{E_{tot}^{2}}}
\end{equation}
\subsubsection{Relativistischer Impuls}
\begin{equation}
\vec{p} = \gamma m \vec{v}
\end{equation}
\begin{equation}
\gamma \equiv \dfrac{1}{\displaystyle\sqrt{1 - \beta^2}} \qquad \text{Lorentzfaktor}
\end{equation}
\subsubsection{Masse-Energie Äquivalenz}
\begin{equation}
E = mc^2
\end{equation}
\begin{equation}
E_{tot} = \sqrt{c^2p^2 + m^2c^4}
\end{equation}
\subsubsection{Kinetische Energie}
Relativistisch:
\begin{equation}
E = mc^2 + E_{kin} \wedge E = \gamma mc^2 \quad\Rightarrow\quad E_{kin} = mc^2(\gamma - 1)
\end{equation}
Klassisch
\begin{equation}
E_{kin} = \dfrac{1}{2} m v^2
\end{equation}
\subsubsection{Potentielle Energie}
\begin{equation}
E_{pot} = m g h
\end{equation}
\subsection{Arbeit}
Die Arbeit, die eine Kraft an einem Körper leistet, ist gleich dem Produkt der Komponente der Kraft längs der Verschiebung und der Verschiebung ($\vartheta$ ist der Winkel zwischen Kraft und Richtung).
\begin{equation}
W = F \Delta x \cos \vartheta
\end{equation}
\subsubsection{Arbeit und potentielle Energie}
\begin{equation}
\Delta E_{pot} = -W
\end{equation}
Diese Gleichung gilt allgemein für \textbf{konservative} Kräfte.
\subsubsection{Bewegung in mehr Dimensionen}
Gegeben sei ein Vektorfeld $\vec{F} = \vec{F}(\vec{r})$ und zwei Punkte $\vec{r_1}, \vec{r_2}$. Die geleistete Arbeit entlang einer differentiellen Strecke $\d W$ ist gleich $ \vec{F}(\vec{r}) \cdot \d \vec{r}$
\begin{equation}
W_{12} = \int_{\vec{r_1}}^{\vec{r_2}} \d W = \int_{\vec{r_1}}^{\vec{r_2}} \vec{F}(\vec{r}) \d \vec{r}
\end{equation}
Dieses Integral hängt vom Weg zwischen $\vec{r_1}$ und $\vec{r_2}$ ab!
\subsection{Beziehung zwischen Kraft und potentieller Energie}
\subsubsection{Der Gradient}
\begin{equation}
\nabla \equiv 
\begin{pmatrix}
\pdiff{}{x} \\[0.3em]
\pdiff{}{y} \\[0.3em]
\pdiff{}{z}
\end{pmatrix}
\end{equation}
Es gilt:
\begin{equation}
\d f = \nabla f \d \vec{r}
\end{equation}
\begin{equation}
\vec{F} = -\nabla E_{pot}
\end{equation}
\subsection{Arbeit-Energie-Theorem}
Die Arbeit die an einem Körper zwischen zwei Punkten geleistet wird ist:
\begin{equation}
W_{12} = \int_{r_1}^{r_2} \vec{F} \d \vec{r} = \frac{1}{2}m\vec{v_2}^2 - \frac{1}{2}m \vec{v_1}^2
\end{equation}
\subsubsection{Die Fluchtgeschwindigkeit}
Hier ist die wirkende Kraft die Gravitationskraft. Die Bahnkurve des Körpers hat keinen Einfluss, die Arbeit hängt nur von der radialen Bewegung des Körpers ab. Somit ergibt sich:
\begin{equation}
W_{12} = \frac{1}{2}m\vec{v_2}^2 - \frac{1}{2}m \vec{v_1}^2 = G m_E m \left( \frac{1}{r_2} - \frac{1}{r_1} \right)
\end{equation}
Nun wird abgeschätzt: $v_2 \to 0, r_2 \to \infty$, was zu folgender Gleichung führt:
\begin{equation}
\vec{v_1}^2 = 2 g r_1
\end{equation}
\subsection{Allgmeine potentielle Energie der Gravitationskraft}
\begin{equation}
E_{pot}(r) = - G \frac{m_e m}{r}
\end{equation}