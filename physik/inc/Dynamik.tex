% !TeX root = ../Formelsammlung.tex
\section{Dynamik}
\subsection{Der lineare Impuls}
\begin{equation}
\vec{p} = m \vec{v}
\end{equation}
\subsection{Newtons Gesetze}
\subsubsection{Trägheit}
Für isolierte Systeme gilt:
\begin{equation}
\vec{p}_{tot} = \text{konst.} \Rightarrow \diff{\vec{p}_{tot}}{t} = 0
\end{equation}
Enthält das System nur einen Körper so folgt:
\begin{equation}
\diff{\vec{p}}{t} = \diff{(m \vec{v})}{t} = m \diff{\vec{v}}{t} = 0 \quad\Rightarrow\quad \vec{v}(t) = \text{konst.} \quad\Rightarrow\quad \vec{a}(t) = 0
\end{equation}
D. h. ein isolierter Körper bewegt sich gleichförmig.
\subsubsection{Aktionsprinzip}
\begin{equation}
\vec{F} \equiv \diff{\vec{p}}{t} \quad\Rightarrow\quad \vec{F} = m \vec{a}(t)
\end{equation}
\subsubsection{Aktion = Reaktion}
Wir betrachten ein System mit zwei Körpern A und B:
\begin{gather}
\vec{p}_{tot} = \vec{p}_A + \vec{p}_B = \text{konst.} \quad\Rightarrow\quad
\diff{\vec{p}_{tot}}{t} = \diff{\vec{p}_A}{t} + \diff{\vec{p}_B}{t} = 0 \quad\Rightarrow \notag\\
\vec{F}_A + \vec{F}_B = 0 \quad\Rightarrow\quad
\vec{F}_A = - \vec{F}_B
\end{gather}
\subsection{Raketenantrieb}
Wir definieren folgende Grössen
\begin{itemize}
\item $v(t)$ Geschwindigkeit bezüglich festem Koordinatensystem.
\item $u$ Konstante Ausstossgeschwindigkeit des Gases \textit{relativ zur Rakete}. Es gilt $u > 0$.
\item $M(t)$ Gesamtmasse der Rakete zum Zeitpunkt $t$.
\end{itemize}
Zum Zeitpunkt $t$ hat die Rakete einen Impuls von $p(t) = M(t)v(t)$. Zur Zeit $t^\prime = t + \mathrm{d}t$ hat sie eine Masse von $M(t) - \d m$ und eine Geschwindigkeit von $v(t) + \d v$. Es für den Impuls gilt:
\begin{equation}
\begin{split}
p(t^\prime) &= M(t)v(t) + M(t)\d v - v(t) \d m - \d m \d v + v(t) \d m - u \d m \\
	 &\approx  M(t)v(t) + M(t)\d v - u \d m 
\end{split}
\end{equation}
Aus der Impulserhaltung folgt
\begin{equation}
\begin{split}
p(t^\prime) - p(t)	& \approx M(t)v(t) + M(t)\d v - u \d m - M(t)v(t) \\
					& = M(t)\d v - u \d m \\
					& \equiv 0
\end{split}
\end{equation}
\begin{equation}
M(t) \d v = u \d m \quad\Rightarrow\quad M(t) \diff{v}{t} = u \diff{m}{t}
\end{equation}
\begin{equation}
F = u \diff{m}{t}
\end{equation}
Durch Integration erhalten wir:
\begin{equation}
v = u \ln \left(\dfrac{1}{1 - m/M_0} \right)
\end{equation}
wobei m(t) die Gesamtmasse \textbf{des ausgestossenen Gases} zur Zeit t beschreibt.
\subsection{Harmonische Schwingungen}
\begin{equation}
\begin{aligned}
&x(t) = A \sin \left( \omega t + \delta \right) \\
&v(t) = A \omega \cos \left( \omega t + \delta \right) \\
&a(t) = -A \omega ^2 \sin \left( \omega t + \delta \right) = - \omega ^2 x(t)
\end{aligned} \qquad
T = \dfrac{2 \pi}{\omega}, \quad f = \nu = \dfrac{1}{T}
\end{equation}
\subsubsection{Die Differentialgleichung der harmonischen Bewegung}
\begin{equation}
F(t) = -kx(t), \quad \text{wobei hier} \enspace k = m \omega ^2
\end{equation}
\begin{equation}
\multidiff{x}{t}{2} + \dfrac{k}{m}x = 0
\end{equation}
Setzt man $x(t) = A \sin \left( \omega t + \delta \right)$ als Lösung ein, erhält man
\begin{equation}
\omega = \sqrt{\dfrac{k}{m}} \quad\Rightarrow\quad T = 2\pi\sqrt{\dfrac{m}{k}}
\end{equation}\\
wobei k die Rückstellkraftkonstante (Proportionalitätsfaktor zwischen Verschiebung und Rückstellkraft) ist.
\subsection{Gravitation}
\begin{equation}
\vec{F}_{12} = -\dfrac{G m_1 m_2}{r_{12}^2} \cdot \dfrac{\vec{r}_{12}}{r_{12}}
\end{equation}
\subsection{Drehimpuls und -moment}
Drehimpuls: $$\vec{L} \equiv \vec{r} \times \vec{p} \equiv m \left( \vec{r} \times \vec{v} \right)$$ 
Drehmoment: $$\vec{M} \equiv \vec{r} \times \vec{F} = \diff{\vec{L}}{t}$$ 
Wirkt kein Drehmoment bleibt der Drehimpuls erhalten.